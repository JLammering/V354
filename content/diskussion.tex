\newpage

\section{Diskussion}
\label{sec:Diskussion}

Eine kleine Abweichung ist bei allen Mess- und Theoriewerten zu erwarten,
da die benutzten Spulen und Kondensatoren als ideal angenommen wurden, was
aber nicht zutrifft.

\subsection{Schwingfall eines RCL-Schwingkreises}

Aus den Messergebnissen wird ein effektiver Widerstand
$R_\text{eff} = \SI{131.4(7)}{\ohm}$ berechnet. Das entspricht einer
prozentualen Abweichung von

\begin{equation}
  1 - \frac{R_\text{eff}}{R} = \SI{80.7}{\percent}
\end{equation}
von dem Widerstand $R$, der in den Schaltkreis eingebaut wird.
Diese große Abweichung lässt sich unter anderem dadurch erklären, dass
der Exponent der Ausgleichsexponentialfunktionen sehr einfach durch kleine
Fehler in den Messdaten für die Minima und Maxima der Schwingung verfälscht
wird. Solche Messfehler entstehen vor allem bei den kleinen Extrema, da
dort die unerwünschten Oberwellen der Stromquelle einen großen Einfluss
haben.


\subsection{Aperiodischer Grenzfall eines RCL-Schwingkreises}

Der gemessene Widerstand in dieser Messreihe beträgt $R_\text{eff} =
\SI{1.9}{\kilo\ohm}$, während der Theoriewert
$R = \SI{5.7(2)}{\kilo\ohm}$ beträgt.
Das entspricht einer prozentualen Abweichung von

\begin{equation}
  1 - \frac{R_\text{eff}}{R} = \SI{66.7}{\percent}.
\end{equation}
Diese große Abweichung lässt sich dadurch erklären, dass die minimale
Überschwingung, die unmittelbar vor dem aperiodischen Grenzfall vorhanden ist
am Oszilloskop kaum erkennbar ist, sodass der Widerstand nur sehr grob
gemessen werden kann.
Außerdem verringern Kabel und Anschlüsse den gemessenen Widerstand, da ihre
Innenwiderstände ebenfalls die Schwingung dämpfen und somit den Grenzfall
beeinflussen.


\subsection{Spannung der erzwungenen Schwingung}

In Abbildung \ref{fig:Theo} sind die Messwerte und die Theoriekurve der
Kondensatorspannung $U_C$ aufgetragen.

\newpage

\begin{figure}[h]
  \centering
  \includegraphics{plotc2.pdf}
  \caption{Messwerte und Theoriekurve der erzwungenen Schwingung.}
  \label{fig:Theo}
\end{figure}

Auffällig ist, dass die Theoriekurve ein deutlich größeres Maximum als der
Graph der Messwerte besitzt. Da die Grenzfrequenzen im Vergleich dazu
sehr nahe aneinander liegen mit Verhältniswerten von

\begin{align}
  \frac{\nu_\text{res,a}}{\nu_\text{res,t}} & = 0.972 &
  \frac{\nu_\text{res,b}}{\nu_\text{res,t}} & = 0.973
\end{align}
und die Frequenzen, die die Breite des Peaks ausmachen, sowie die berechneten
Bauteilwerte sich jeweils stark von den Theoriewerten unterscheiden,
liegen auch die jeweiligen Endergebnisse für die Güte weit auseinander.
Die Fehler der genäherterten Bauteilwerte in dem Fit werden nicht in die
Rechnungen einbezogen, da sie zum Teil $10^4$-mal so groß wie die
Werte sind.
Die prozentualen Abweichungen zwischen den aus den Messwerten berechneten Güten
und dem Theoriewert betragen

\begin{align}
  1 - \frac{q_a}{q_\text{t}} & = \SI{56.2}{\percent} &
  1 - \frac{q_b}{q_\text{t}} & = \SI{55.2}{\percent}.
\end{align}
Die großen Unterschiede in den Resonanz-Peaks der beiden Werte treten vor allem
dadurch auf, dass ein sehr kleiner Widerstand $R = \SI{67.2(2)}{\ohm}$
benutzt wurde. Minimale Widerstände in den Kabeln und Schaltungen
haben so deutlich größere Auswirkungen.
Außerdem ist die Spannung der Resonanzfrequenz bei kleineren Widerständen
deutlich höher als bei größeren, wodurch die Auswirkung der nicht mit
einberechneten Widerstände umso größer ist.
Die Bestimmung der Güte über die Spannung eignet sich nur bei hohen
Widerständen.

\subsection{Phasenverschiebung der erzwungenen Schwingung}

Bei der letzten Messung ist die Abweichung zwischen Mess- und Theoriewert
mit

\begin{equation}
  1-\frac{q_\text{c}}{q_\text{t}} =
  \SI{17.5}{\percent}
\end{equation}
gering.
Die Bestimmung der Güte über die Phasenverschiebung eignet sich also sehr gut.
