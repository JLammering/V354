\section{Auswertung}
\label{sec:Auswertung}

\subsection{Schwingfall eines RCL-Schwingkreises}


\subsubsection{Messwerte}

In der ersten Messung wird mit Hilfe des Oszilloskops der Schwingfall eines
RCL-Schwingkreises aufgezeichnet. Die Minima und Maxima der Messwerte der
gedämpften Schwingung sind in Tabelle \ref{tab:Schwingfall} dargestellt.

\begin{table}[h]
  \centering
  \begin{tabular}{c c c c}
    \toprule
    \multicolumn{2}{c}{Minima} & \multicolumn{2}{c}{Maxima} \\
    $t/\si{\micro\second}$ & $U/\si{\V}$
    & $t/\si{\micro\second}$ & $U/\si{\V}$  \\
    \midrule
    36.6 & 14 & 18.4 & 200 \\
    74.6 & 26 & 56.2 & 186 \\
    111.8 & 38 & 92.8 & 176 \\
    149.8 & 46 & 130.8 & 164 \\
    188.6 & 54 & 169.0 & 156 \\
    225.6 & 60 & 206.8 & 150 \\
    263.8 & 66 & 245.6 & 142 \\
    302.2 & 70 & 283.8 & 136 \\
    339.8 & 74 & 321.6 & 132 \\
    377.6 & 78 & 357.0 & 126 \\
    416.6 & 80 & 396.6 & 124 \\
    453.6 & 82 & 434.4 & 122 \\
    \bottomrule
  \end{tabular}
  \caption{Maxima und Minima des Schwingfalls}
  \label{tab:Schwingfall}
\end{table}

Die Offsetspannung beträgt bei jedem Messwert

\begin{equation}
  U_\text{offset} = \SI{96.9}{\V}.
\end{equation}
In Abbildung \ref{fig:Schwingfall} ist unter anderem der Graph der Messwerte
mit einberechnetem Offset abgebildet.

\subsubsection{Rechnung}

Zunächst werden mit der Funktion $curvefit$ aus $scipy.optimize$ zwei
Ausgleichsfunktionen bestimmt, die durch die Maxima und Minima
verlaufen. Diese haben wie schon in der Theorie erwähnt jeweils die Form einer
Exponentialfunktion.
Die obere Einhüllende, die durch die Maxima verläuft, entspricht mit

\begin{equation}
  U_1(t) = \SI{109.33}{\V} \symup{e}^{-2\symup{\pi} \cdot 568.79\cdot t}
\end{equation}
etwa dem Betrag der unteren Einhüllenden

\begin{equation}
  U_2(t) = \SI{-96.52}{\V} \symup{e}^{-2\symup{\pi} \cdot 677.96\cdot t}.
\end{equation}
Dies ist ebenfalls in Abbildung \ref{fig:Schwingfall} erkennbar.

\begin{figure}[h]
  \centering
  \includegraphics{plota.pdf}
  \caption{Messwerte des Schwingfalles und Einhüllende $U_1$ und $U_2$.}
  \label{fig:Schwingfall}
\end{figure}

Aus den Exponenten der Exponentialfunktionen wird das arithmetische Mittel
gebildet. Es folgt

\begin{equation}
  \mu = 623.375
\end{equation}
und daraus über die Formel

\begin{equation}
  T_\text{ex} = \frac{2L}{R} = \frac{1}{2\symup{\pi}\mu}
\end{equation}
der effektive Dämpfungswiderstand

\begin{equation}
  R_\text{eff} = \SI{131.447}{\ohm}
\end{equation}
und die Abklingdauer

\begin{equation}
  T_\text{ex} = \SI{0.0002553}{\second}.
\end{equation}
Der benutzte Dämpfungswiderstand beträgt wie schon in der Durchführung erwähnt

\begin{equation}
  R =  .
\end{equation}


\subsection{Aperiodischer Grenzfall eines RCL-Schwingkreises}

In der zweiten Messung wird ein variabler Widerstand so verändert, dass
sich an dem RCL-Schwingkreis der aperiodische Grenzfall einstellt.
Für den effektiven variablen Widerstand wird dafür der Wert

\begin{equation}
  R_\text{eff} = \SI{1.9}{\kilo\ohm}
\end{equation}
gemessen.
Der Theoriewert für den Dämpfungswiderstand bei gegebener Kapazität $C$ und
Induktivität $L$ ist

\begin{equation}
  R = \sqrt{\frac{4L}{C}}.
\end{equation}


\subsection{Kondensatorspannung und Phasenverschiebung
einer erzwungenen Schwingung}

In der dritten Messreihe wird sowohl die Kondensatorspannung als auch die
Phasenverschiebung der erzwungenen Schwingung im RCL-Schwingkreis in
Abhängigkeit von der Frequenz gemessen. In Abbildung \ref{fig:plotc1} sind
die Messwerte der Spannung aufgetragen. Aus den Werten für die
Resonanzfrequenz

\begin{equation}
  \nu_\text{res} = \SI{26000}{\hertz}
\end{equation}
und den Grenzen des Resonanz-Peaks

\begin{align}
  \nu_- = & \SI{25600}{\V} & \nu_+ = & \SI{27000}{\V}, \\
\end{align}
die jeweils an der Kondensatorspannung

\begin{equation}
  U_\text{+-} = \frac{SI{186}{\V}}{\sqrt{2}} = SI{131.5}{\V}
\end{equation}
liegen, kann nun die Güte $q$ des Schwingkreises bestimmt werden.
Aus der Formel

\begin{equation}
  q = \frac{\nu_\text{res}}{\nu_+ - \nu_-}
\end{equation}
folgt für die Güte

\begin{equation}
  q = 1.857 .
\end{equation}




\begin{figure}[h]
  \centering
  \includegraphics{plotc1.pdf}
  \caption{Plotc.}
  \label{fig:plotc1}
\end{figure}
