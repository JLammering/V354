\section{Auswertung}
\label{sec:Auswertung}

\subsection{Schwingfall eines RCL-Schwingkreises}


\subsubsection{Messwerte}

In der ersten Messung wird mit Hilfe des Oszilloskops der Schwingfall eines
RCL-Schwingkreises aufgezeichnet. Die Minima und Maxima der Messwerte der
gedämpften Schwingung sind in Tabelle \ref{tab:Schwingfall} dargestellt.

\begin{table}[h]
  \centering
  \begin{tabular}{c c c c}
    \toprule
    \multicolumn{2}{c}{Minima} & \multicolumn{2}{c}{Maxima} \\
    $t/\si{\micro\second}$ & $U/\si{\V}$
    & $t/\si{\micro\second}$ & $U/\si{\V}$  \\
    \midrule
    36.6 & 14 & 18.4 & 200 \\
    74.6 & 26 & 56.2 & 186 \\
    111.8 & 38 & 92.8 & 176 \\
    149.8 & 46 & 130.8 & 164 \\
    188.6 & 54 & 169.0 & 156 \\
    225.6 & 60 & 206.8 & 150 \\
    263.8 & 66 & 245.6 & 142 \\
    302.2 & 70 & 283.8 & 136 \\
    339.8 & 74 & 321.6 & 132 \\
    377.6 & 78 & 357.0 & 126 \\
    416.6 & 80 & 396.6 & 124 \\
    453.6 & 82 & 434.4 & 122 \\
    \bottomrule
  \end{tabular}
  \caption{Maxima und Minima des Schwingfalls}
  \label{tab:Schwingfall}
\end{table}

Die Offsetspannung beträgt bei jedem Messwert

\begin{equation}
  U_\text{offset} = \SI{96.9}{\V}.
\end{equation}

In Abbildung \ref{fig:Schwingfall} ist unter anderem der Graph der Messwerte
mit einberechnetem Offset abgebildet.

\subsubsection{Rechnung}

Zunächst werden mit der Funktion $curvefit$ aus $scipy.optimize$ zwei
Ausgleichsfunktionen bestimmt, die durch die Maxima und Minima
verlaufen. Diese haben wie schon in der Theorie erwähnt jeweils die Form einer
Exponentialfunktion.
Die obere Einhüllende $I_1(t)$, die durch die Maxima verläuft, entspricht mit

\begin{equation}
  I_1(t) = A \symup{e}^{zt}
\end{equation}
etwa dem Betrag der unteren Einhüllenden

\begin{equation}
  I_2(t) = - B \symup{e}^{rt}.
\end{equation}


\begin{figure}[h]
  \centering
  \includegraphics{plota1.pdf}
  \caption{Messwerte des Schwingfalles und Einhüllende $I_1$ und $I_2$.}
  \label{fig:Schwingfall}
\end{figure}

\begin{figure}[h]
  \centering
  \includegraphics{plotc.pdf}
  \caption{Plotc.}
  \label{fig:plotc}
\end{figure}

\subsection{Aperiodischer Grenzfall eines RCL-Schwingkreises}

\subsubsection{Messwerte}
