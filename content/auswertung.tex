\newpage
\section{Auswertung}
\label{sec:Auswertung}

\subsection{Schwingfall eines RCL-Schwingkreises}

\subsubsection{Messwerte}

In der ersten Messung wird mit Hilfe des Oszilloskops der Schwingfall eines
RCL-Schwingkreises aufgezeichnet. Die Minima und Maxima der Messwerte der
gedämpften Schwingung sind in Tabelle \ref{tab:Schwingfall} dargestellt.

\begin{table}[h]
  \centering
  \begin{tabular}{c c c c}
    \toprule
    \multicolumn{2}{c}{Minima} & \multicolumn{2}{c}{Maxima} \\
    $t/\si{\micro\second}$ & $U/\si{\V}$
    & $t/\si{\micro\second}$ & $U/\si{\V}$  \\
    \midrule
    36.6 & 14 & 18.4 & 200 \\
    74.6 & 26 & 56.2 & 186 \\
    111.8 & 38 & 92.8 & 176 \\
    149.8 & 46 & 130.8 & 164 \\
    188.6 & 54 & 169.0 & 156 \\
    225.6 & 60 & 206.8 & 150 \\
    263.8 & 66 & 245.6 & 142 \\
    302.2 & 70 & 283.8 & 136 \\
    339.8 & 74 & 321.6 & 132 \\
    377.6 & 78 & 357.0 & 126 \\
    416.6 & 80 & 396.6 & 124 \\
    453.6 & 82 & 434.4 & 122 \\
    \bottomrule
  \end{tabular}
  \caption{Maxima und Minima des Schwingfalls.}
  \label{tab:Schwingfall}
\end{table}

Die Offsetspannung beträgt bei jedem Messwert

\begin{equation}
  U_\text{offset} = \SI{96.9}{\V}.
\end{equation}
In Abbildung \ref{fig:Schwingfall} ist unter anderem der Graph der Messwerte
mit einberechnetem Offset abgebildet.

\subsubsection{Rechnung}

Zunächst wird eine Ausgleichsfunktionen bestimmt, die durch die Maxima und
Minima verläuft.
Diese hat mit

\begin{equation}
  U(t) = \pm a \cdot \symup{e}^{-2\symup{\pi}\cdot b \cdot t}
\end{equation}
die Form einer Exponentialfunktion, wie bereits in der Theorie erwähnt.
Die berechneten Koeffizienten der Einhüllenden lauten

\begin{align}
  a & = \SI{104(4)}{\V} \\
  b & = \SI{6.3(4)e02}{\per\second}.
\end{align}
Die Funktion ist in Abbildung \ref{fig:Schwingfall} dargestellt.

\begin{figure}[h]
  \centering
  \includegraphics{plota.pdf}
  \caption{Messwerte des Schwingfalles und Einhüllende $U_1$ und $U_2$.}
  \label{fig:Schwingfall}
\end{figure}

Über die Formel

\begin{equation}
  T_\text{ex} = \frac{2L}{R} = \frac{1}{2\symup{\pi}b}
\end{equation}
folgt nun der effektive Dämpfungswiderstand

\begin{equation}
  R_\text{eff} = \SI{133(9)}{\ohm}
\end{equation}
und die Abklingdauer

\begin{equation}
  T_\text{ex} = \SI{0.25(2)}{\milli\second} .
\end{equation}
Der benutzte Dämpfungswiderstand beträgt, wie bereits in der Durchführung
erwähnt,

\begin{equation}
  R = \SI{682(1)}{\ohm} .
\end{equation}


\subsection{Aperiodischer Grenzfall eines RCL-Schwingkreises}

In der zweiten Messung wird ein variabler Widerstand so verändert, dass
sich an dem RCL-Schwingkreis der aperiodische Grenzfall einstellt.
Für den effektiven variablen Widerstand wird dafür der Wert

\begin{equation}
  R_\text{eff} = \SI{1.9}{\kilo\ohm}
\end{equation}
eingestellt.
Der Theoriewert für den Dämpfungswiderstand bei gegebener Kapazität $C$ und
Induktivität $L$ ist

\begin{equation}
  R = \sqrt{\frac{4L}{C}} = \SI{5.7(2)}{\kilo\ohm}.
\end{equation}


\subsection{Spannung und Phasenverschiebung
einer erzwungenen Schwingung}

\subsubsection{Messwerte}

In der dritten Messreihe wird sowohl die Kondensatorspannung $U_C$ als auch die
Phasenverschiebung $\varphi$ der erzwungenen Schwingung im RCL-Schwingkreis in
Abhängigkeit von der Frequenz $\nu$ gemessen.
In den Tabellen \ref{tab:Messung3a} und \ref{tab:Messung3b}
sind die aufgenommenen Messwerte abgebildet.
Die Phasenverschiebung ist nicht als Winkel, sondern als Zeitdifferenz $t_D$
dargestellt.

\begin{table}[h]
  \centering
  \begin{tabular}{c c c}
    \toprule
    $\nu/\si{\hertz}$ & $t_D/\si{\micro\second}$ & $U_C/\si{\volt}$ \\
    \midrule
    4.26 & 0 & 5.2 \\
    100 & 0 & 10 \\
    200 & 0 & 10.1 \\
    500 & 0 & 10 \\
    1000 & 0 & 10 \\
    2000 & 0 & 10 \\
    4000 & 0 & 10.1 \\
    6000 & 0 & 10.5 \\
    8000 & 0 & 10.8 \\
    10000 & 0 & 11 \\
    12000 & 0 & 12 \\
    14000 & 0 & 12.6 \\
    16000 & 0 & 14.8 \\
    18000 & 0 & 17.6 \\
    20000 & 0 & 22.4 \\
    21000 & 0 & 26 \\
    22000 & 0 & 31.2 \\
    23000 & 0 & 40 \\
    23500 & 0 & 45.6 \\
    24000 & 0 & 54.4 \\
    24500 & 0 & 66.4 \\
    25000 & 2 & 89 \\
    25200 & 2 & 100 \\
    25400 & 2.5 & 115 \\
    25600 & 3 & 132 \\
    25800 & 4 & 153 \\
    26000 & 5 & 173 \\
    26100 & 6 & 181 \\
    26200 & 8 & 186 \\
    \bottomrule
  \end{tabular}
  \caption{Messwerte der erzwungenen Schwingung.}
  \label{tab:Messung3a}
\end{table}

\begin{table}[h]
  \centering
  \begin{tabular}{c c c}
    \toprule
    $\nu/\si{\hertz}$ & $t_D/\si{\micro\second}$ & $U_C/\si{\volt}$ \\
    \midrule
    26300 & 10 & 186 \\
    26400 & 11 & 184 \\
    26500 & 12 & 177 \\
    26600 & 13 & 169 \\
    26700 & 14 & 160 \\
    26800 & 15 & 149 \\
    26900 & 15 & 139 \\
    27000 & 15 & 130 \\
    27100 & 16 & 120 \\
    27200 & 16 & 112 \\
    27300 & 16 & 105 \\
    27500 & 17 & 92 \\
    27700 & 17 & 82 \\
    28000 & 17.5 & 70 \\
    28500 & 17.5 & 56 \\
    29000 & 17.5 & 46 \\
    30000 & 17 & 34 \\
    31000 & 17 & 28 \\
    33000 & 16.5 & 20 \\
    35000 & 15 & 16 \\
    40000 & 13 & 8.8 \\
    45000 & 11 & 6.4 \\
    \bottomrule
  \end{tabular}
  \caption{Messwerte der erzwungenen Schwingung.}
  \label{tab:Messung3b}
\end{table}


\subsubsection{Berechnung der Güte}

In Abbildung \ref{fig:plotc1} sind
die Messwerte der Kondensatorspannung und der zugehörige Fit logarithmisch
aufgetragen. Aus dem Wert für die Resonanzfrequenz

\begin{equation}
  \nu_\text{res,a} = \SI{26271.8}{\hertz}
\end{equation}
und dem Kondensator- und Widerstandswert der Regression

\begin{align}
  C_\text{fit} & = \SI{1.714e-09}{\farad} & R_\text{fit} & = \SI{190.3}{\ohm}
\end{align}
folgt über die Gleichung

\begin{equation}
  q_\text{a} = \frac{1}{\omega_\text{res,a}C_\text{fit}R_\text{fit}}
\end{equation}
die Güte

\begin{equation}
  q_\text{a} = 18.57.
\end{equation}

\newpage

\begin{figure}[h]
  \centering
  \includegraphics{plotc1.pdf}
  \caption{Messwerte und Fit der Resonanzkurve.}
  \label{fig:plotc1}
\end{figure}

In Abbildung \ref{fig:plotc3} sind die Messwerte des Resonanzpeaks dargestellt
und durch zwei Geraden angenähert. Aus der Resonanzfrequenz

\begin{equation}
  \nu_\text{res,b} = \SI{263(6)e02}{\hertz}
\end{equation}
und der Resonanzpeakbreite

\begin{equation}
  \nu_+ - \nu_- = \SI{13(4)e02}{\hertz}
\end{equation}
die jeweils an der Kondensatorspannung

\begin{equation}
  U_{+-} = \frac{\SI{1.9(5)e02}{\V}}{\sqrt{2}} = \SI{1.4(4)e02}{\V}
\end{equation}
liegen, kann nun die Güte $q$ des Schwingkreises bestimmt werden.
Aus der Formel

\begin{equation}
  q_\text{b} = \frac{\nu_\text{res}}{\nu_+ - \nu_-}
\end{equation}
folgt für die Güte

\begin{equation}
  q_\text{b} = \num{19(5)}.
\end{equation}
Der Theoriewert der Güte hat mit der Grenzfrequenz

\begin{equation}
  \nu_\text{res,t} = \SI{27.03(8)e02}{\hertz}
\end{equation}
und den benutzten Kondensator- und Widerstandswerten den Wert

\begin{equation}
  q_\text{t} = \num{42.4(2)} .
\end{equation}

\begin{figure}[h]
  \centering
  \includegraphics{plotc3.pdf}
  \caption{Messwerte am Resonanzpeak und lineare Regression.}
  \label{fig:plotc3}
\end{figure}

\subsubsection{Berechnung der Phasenverschiebung}

Mit Hilfe der Formel

\begin{equation}
  \varphi = t_D \cdot 2\symup{\pi}f
\end{equation}
wird für jeden Differenzwert $t_D$ die Phasenverschiebung $\varphi$ bestimmt.
Die Phasenverschiebung für $\varphi > \SI{0}{\degree}$ ist in Abbildung
\ref{fig:Phasegros} gegen die Frequenz $\nu$ aufgetragen.

\newpage

\begin{figure}[h]
  \centering
  \includegraphics{plotd2.pdf}
  \caption{Phasenverschiebung zwischen $0$ und $\symup{\pi}$.}
  \label{fig:Phasegros}
\end{figure}

Der Bereich um $\varphi = \SI{90}{\degree}$ wird durch eine Ausgleichsgerade
angenähert. Diese ist in Abbildung \ref{fig:Phaseklein} abgebildet.
Aus diesem Graphen erhält man die Resonanzfrequenz

\begin{equation}
  \nu_\text{res,c} = \SI{263(15)e2}{\hertz}
\end{equation}
und die Werte

\begin{align}
  \nu_1 = & \SI{259(15)e2}{\hertz} & \nu_2 = & \SI{267(16)e2}{\hertz} \\
\end{align}
an den Stellen

\begin{align}
  \varphi_1 = & \SI{45}{\degree} & \varphi_2 = & \SI{135}{\degree}. \\
\end{align}
Für die Güte folgt

\begin{equation}
  q_\text{c} = \num{35(3)}.
\end{equation}

\begin{figure}[h]
  \centering
  \includegraphics{plotd3.pdf}
  \caption{Phasenverschiebung um $\varphi = \SI{90}{\degree}$ und lineare
  Regression.}
  \label{fig:Phaseklein}
\end{figure}
